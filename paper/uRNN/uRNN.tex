\documentclass{article} % For LaTeX2e
\usepackage{iclr2016_conference,times}
\usepackage{hyperref}
\usepackage{url}
\usepackage{graphicx}
\usepackage{amsmath,amsfonts,amssymb}
\usepackage{mathtools}
\usepackage{amsthm}
\usepackage{color}

\title{\scalebox{0.95}{Unitary Evolution Recurrent Neural Networks}}


\author{Martin Arjovsky \thanks{Indicates first authors. Ordering determined by coin flip.} \\
Universidad de Buenos Aires\\
\texttt{marjovsky@dc.uba.ar} \\
\And
Amar Shah$^*$ \\
Cambridge University \\
\texttt{as793@cam.ac.uk} \\
\AND
Yoshua Bengio \\
Universit\'e de Montr\'eal, CIFAR Senior Fellow
% YB: I prefer to not put my e-mail address on papers
}

% The \author macro works with any number of authors. There are two commands
% used to separate the names and addresses of multiple authors: \And and \AND.
%
% Using \And between authors leaves it to \LaTeX{} to determine where to break
% the lines. Using \AND forces a linebreak at that point. So, if \LaTeX{}
% puts 3 of 4 authors names on the first line, and the last on the second
% line, try using \AND instead of \And before the third author name.

\newcommand{\fix}{\marginpar{FIX}}
\newcommand{\new}{\marginpar{NEW}}
\newcommand{\matr}[1]{\mathbf{#1}}
\newcommand\RR{\mathbb{R}}
\newcommand\CC{\mathbb{C}}
\newtheorem{lemma}{Lemma}
\newcommand\norm[1]{\left\lVert#1\right\rVert}
%\iclrfinalcopy % Uncomment for camera-ready version

\begin{document}


\maketitle

\begin{abstract}
Recurrent neural networks (RNNs) are notoriously difficult to train. When the eigenvalues of the hidden to hidden weight matrix
deviate from absolute value 1, optimization becomes difficult due to the well studied issue of {\it{vanishing}} and {\it{exploding}} gradients, especially when trying to learn long-term dependencies.
To circumvent this problem, we propose a new architecture that learns a unitary weight matrix, with eigenvalues
of absolute value exactly 1. The challenge we address is that of parametrizing unitary matrices in a way that does not require expensive computations (such as eigendecomposition) after each weight update. We construct an expressive unitary weight matrix by composing several structured matrices that act
as building blocks with parameters to be learned. Optimization with this parameterization becomes feasible only when considering hidden
states in the complex domain. We demonstrate the potential of this architecture by achieving state of the art
results in several hard tasks
involving very long-term dependencies.

\end{abstract}

\section{Introduction}
Deep Neural Networks have shown remarkably good performance on a wide range of complex data problems 
including speech recognition \citep{Hinton2012}, image recognition \citep{Krizhevsky2012} and natural 
language processing \citep{Collobert2011}. However, training very deep models remains a difficult task. 
The main issue surrounding the 
training of deep networks is the {\it{vanishing}} and {\it{exploding}} gradients problems
introduced by~\citet{Hochreiter91-small} and shown by~\citet{Yoshua94} to be necessarily arising when trying to learn
to reliably store bits of information in any parametrized dynamical system.
If gradients propagated back through a network vanish, the credit assignment role of backpropagation
is lost, as information about small changes in states in the far past has no influence on future states.
If gradients explode, gradient based optimization algorithms struggle to 
traverse down a cost surface, because gradient-based optimization relies on smoothness of
the objective: it assumes that a small change in the parameters will yield a small change in the objective function,
and this assumption is violated when very large gradients occur. As the number of time steps
considered in the sequence of states grow, the shrinking or expanding effects associated with the state-to-state
transformation at individual time steps can grow exponentially, yielding respectively vanishing or exploding
gradients. See~\citet{Pascanu2013} for a review.

Although the long-term dependencies problem appears intractable in the absolute~\citep{Yoshua94} for
parametrized dynamical systems, several heuristics have been found in recent years to help considerably
reduce its effect, such as the use of self-loops and gating units in the LSTM~\citep{LSTM}
and GRU~\citep{Cho2014a} recurrent architectures.
Recent work also supports the idea of using \textit{orthogonal} weight matrices to assist optimization  
\citep{Saxe2014} \citep{Quoc2015}. A matrix, $\matr{W}$, is orthogonal if 
$\matr{W}^\top \matr{W} = \matr{W} \matr{W}^\top = \matr{I}$. 
Orthogonal matrices have the property that they preserve norm (i.e. $\| \matr{W} h \|_2 = \| h \|_2$)
and hence repeated iterative multiplication of a vector by an orthogonal matrix leaves the norm of the 
vector unchanged.

Let $h_T$ and $h_t$ be the hidden unit vectors for hidden layers $T$ and $t$ of a neural network with 
$T$ hidden layers and $T \gg t$. 
If $C$ is the objective we are trying to minimize, then the {\it{vanishing}} and {\it{exploding}} 
gradient problem refer to the decay or growth of $\frac{\partial C}{\partial h_t}$ as the 
number of layers, $T$, grows. Let $\sigma$ be a pointwise nonlinearity function, and
\begin{equation}
  z_{t+1} = \matr{W}_t h_t + \matr{V}_t x_{t+1}
\label{linoutput}
\end{equation}
\begin{equation}
  h_{t+1} = \sigma (z_{t+1})
\label{nonlinoutput}
\end{equation}

then by the chain rule
\begin{equation}
  \frac{\partial C}{\partial h_t} = \frac{\partial C}{\partial h_T} \frac{\partial h_T}{\partial h_t} 
  = \frac{\partial C}{\partial h_T} \prod_{k=t}^{T-1} \frac{\partial h_{k+1}}{\partial h_k} 
  = \frac{\partial C}{\partial h_T} \prod_{k=t}^{T-1} \matr{D}_{k+1} \matr{W}_k^T ,
\end{equation}

where $\matr{D}_{k+1} = diag(\sigma'(z_{k+1}))$ is the Jacobian matrix of the pointwise nonlinearity.

In the following we define to the norm of a matrix to refer to the spectral radius norm (or operator 2-norm)
and the norm of a vector to mean $L_2$-norm. By definition of the operator norms, 
for any matrices $\matr{A}, \matr{B}$ and vector $v$ we have $\norm{\matr{A}v} \leq \norm{\matr{A}} \norm{v}$ and $\norm{\matr{A}\matr{B}} \leq \norm{\matr{A}} \norm{\matr{B}}$.
If the weight matrices $\matr{W}_k$ are norm preserving (i.e. orthogonal), then we {\bf{prove}}
\begin{equation}
  \norm{ \frac{\partial C}{\partial h_t} } = \norm{ \frac{\partial C}{\partial h_T} 
  \prod_{k=t}^{T-1} \matr{D}_{k+1} \matr{W}_k^T } \leq \norm{\frac{\partial C}{\partial h_T}} 
  \prod_{k=t}^{T-1} \norm{ \matr{D}_{k+1} \matr{W}_k^T } 
  = \norm{ \frac{\partial C}{\partial h_T}} \prod_{k=t}^{T-1} \norm{\matr{D}_{k+1}} .
\label{bound}
\end{equation}

Since $\matr{D}_k$ is diagonal, $\norm{\matr{D}_k} = \max_{j=1, ..., n} |\sigma'(z_k^{(j)})|$,
with $z_k^{(j)}$ the $j$-th pre-activation of the $k$-th hidden layer.
If the absolute value of the derivative $\sigma'$ can take some value $\tau > 1$, then 
this bound is useless, since $\norm{\frac{\partial C}{\partial h_t}} 
\leq \norm{\frac{\partial C}{\partial h_T}} \tau^{T-t}$ which grows exponentially 
in $T$. We therefore cannot effectively bound $\frac{\partial C}{\partial h_t}$ 
for deep networks, resulting potentially in exploding gradients.

In the case $|\sigma'| < \tau < 1$, equation \ref{bound} proves that 
that $\frac{\partial C}{\partial h_t}$ tends to 0 exponentially fast as $T$ grows, 
resulting in guaranteed vanishing gradients. 
This argument makes the rectified linear unit (ReLU) nonlinearity an attractive choice
\citep{Glorot2011, Nair2010}. Unless all the activations are killed at one layer, 
the maximum entry of $\matr{D}_k$ is 1, resulting in
$\norm{\matr{D}_k} = 1$ for all layers $k$. With ReLU nonlinearities, we thus have
\begin{equation}
  \norm{\frac{\partial C}{\partial h_t}} \leq \norm{ \frac{\partial C}{\partial h_T}} 
  \prod_{k=t}^{T-1} \norm{\matr{D}_{k+1}} = \norm{\frac{\partial C}{\partial h_T}}
\label{bound2}
\end{equation}

Most notably, this result holds for a network of arbitrary depth. 
As a consequence of equation \ref{bound2} and under mild conditions, we can show that
for a recurrent neural network with an orthogonal hidden to hidden weight matrix, 
parameter gradients grow at most quadratically with depth as opposed to exponentially
{\color{red}(see Appendix)}. This property makes engineering tricks like gradient clipping no longer
necessary \citep{Pascanu2013}.

To the best of our knowledge, this analysis is a novel contribution and the first time a 
neural network architecture has been mathematically proven to avoid exploding gradients. 

In this paper, we explore the use of orthogonal matrices in recurrent neural networks. 
Section 2 discusses the difficulties of parameterizing real valued orthogonal matrices and how
they are easily alleviated by moving to the \textit{complex} domain. 
A complex valued, norm preserving matrix,
$\matr{U}$, is called a \textit{unitary} matrix and is such that 
$\matr{U}^* \matr{U} = \matr{U} \matr{U}^* = \matr{I}$, where $\matr{U}^*$ is the conjugate transpose
of $\matr{U}$.   

We discuss a novel approach to constructing expressive unitary matrices as the composition of simple
unitary matrices which require at most $\mathcal{O}(n \log n)$ computation and $\mathcal{O}(n)$ memory,
unlike general matrices which require $\mathcal{O}(n^2)$ computation and memory. {\color{red} discuss how 
others have tried to work with complex valued representations with little success... }

Whilst our model uses complex valued matrices and parameters, all implementaion and optimization is 
possible with real numbers. This along with other implementation details are discussed in Section 3.
The potential of the developed model for learning long term dependencies with relatively few parameters is
explored in Section 4.



%Notes: 
%
%- Vanishing and exploding gradients.
%
%- Support for orthogonal matrices. Results of Quoc's stuff with orthogonality.
%
%- Prove lower bound for orthogonal matrices up to jacobian matrix norms.
%
%- Introduce ReLUs as a natural solution to norm of D to be 1. Support for relus, why orthogonality + relus make sense (additive growth at most).
%
%- 

\section{Unitary Evolution RNNs}

The most important feature of unitary and orthogonal matrices for our purpose is that they have eigenvalues
$\lambda_j$ with absolute value 1. The following lemma, proved in \cite{linalgbook}, may shed light on a 
method by which we may construct orthogonal matrices.

\begin{lemma}
  A complex square matrix $\matr{W}$ is unitary if and only if it has an eigendecomposition of the form
  $\matr{W} = \matr{V} \matr{D} \matr{V}^*$. Here, $\matr{V}, \matr{D} \in \mathbb{C}^{n \times n}$ are
  complex matrices, where $\matr{V}$ is unitary, and $\matr{D}$ is a diagonal such that $|\matr{D}_{j,j}|=1$. 
  Furthermore, $\matr{W}$ is a real orthogonal matrix if and only if for every eigenvalue $\matr{D}_{j,j} 
  = \lambda_j$ with eigenvector $v_j$, there is also an eigenvalue $\lambda_k = \overline{\lambda_j}$ with 
  corresponding eigenvector $v_k = \overline{v_j}$ .
\label{lemma}
\end{lemma}

Writing $\lambda_j = e^{i w_j}$ with $w_j \in \RR$, a naive method to learn a unitary matrix would be to
fix a basis of eigenvectors $\matr{V} \in \mathbb{C}^{n \times n}$ and set
\begin{equation} \matr{W} = \matr{V} \matr{D} \matr{V}^{*} , \end{equation}

where $\matr{D}$ is a diagonal such that $\matr{D}_{j,j} = \lambda_j$. 

Lemma \ref{lemma} informs us how to construct a real orthogonal matrix, $\matr{W}$.
We must (i) ensure the columns of $\matr{V}$ come in complex conjugate pairs, $v_k = \overline{v_j}$, and
(ii) tie weights $w_k=-w_j$ in order to achieve $e^{i w_j} = \overline{e^{i w_k}}$. 
Most neural network objective functions are differentiable with respect to the weight matrices,
and consequently $w_j$ may be learned by gradient descent. 

Unfortunately the above approach has undesirable properties. 
Fixing $\matr{V}$ and learning $w$ requires $\mathcal{O}\left(n^2\right)$ memory, 
which is unacceptable given that the number of learned parameters is $\mathcal{O}(n)$. 
Further note that calculating $\matr{V} u$ for an arbitrary vector $u$ 
requires $\mathcal{O}(n^2)$ computation. 
Setting $\matr{V}$ to the identity would satisfy the conditions of the lemma, whilst reducing  
memory and computation requirements to $\mathcal{O}(n)$, however, $\matr{W}$ would remain diagonal, 
and have poor representation capacity.

We propose an alternative strategy to parameterize unitary matrices. 
Since the product of unitary matrices is itself a unitary matrix, our idea involves
composing several simple, parameteric, unitary matrices to construct a single, expressive unitary matrix.
The four types of unitary building blocks considered are 

\begin{itemize}
  \item $\matr{D}$, a diagonal matrix with $\matr{D}_{j,j} = e^{i w_j}$, with parameters $w_j \in \RR$,
  \item $\matr{R} = \matr{I} - 2 \frac{v v^*}{\|v\|^2}$, a reflection matrix in the complex vector 
  $v \in \mathbb{C}^n$, 
  \item $\matr{\Pi}$, a fixed random index permutation matrix, and
  \item $\mathcal{F}$ and $\mathcal{F}^{-1}$, the Fourier and inverse Fourier transforms.
\end{itemize}

Appealingly, $\matr{D}$, $\matr{R}$ and $\matr{\Pi}$ all permit $\mathcal{O}(n)$ storage and 
$\mathcal{O}(n)$ computation for matrix vector products. $\mathcal{F}$ and $\mathcal{F}^{-1}$
require no storage and $\mathcal{O}(n \log n)$ matrix vector multiplication using the Fast Fourier
Transform algorithm. A major advantage of composing unitary matrices of the form listed above, is 
that the number of parameters, memory and computational cost increase almost linearly in the size
of the hidden layer. With such a weight matrix, immensely large hidden layers are feasible to train, 
whilst being impossible in traditional neural networks. 
 
With this in mind, we choose to conisder recurrent neural networks with unitary hidden to hidden
weight matrices in this work. Our claim is that the ability to have large hidden layers where hidden 
states norms are preserved provides a powerful tool for modelling long term dependencies in sequence data. 
Early work by \cite{Yoshua94} suggests that having a large memory may be crucial for solving 
difficult tasks involving long ranging dependencies.

We call any RNN architecture which uses a unitary hidden to hidden matrix a \textit{unitary evolution RNN}
(uRNN). After experimenting with several structures, we settled on the following composition

\begin{equation} \matr{W} = \matr{D}_3 \matr{R}_2 \mathcal{F}^{-1} \matr{D}_2 \matr{\Pi} \matr{R}_1 \mathcal{F} \matr{D}_1 .\end{equation}

Whilst each but the permutation matrix is complex, we parameterize and represent them with real numbers
for implementation purposes. When the final cost is real and differentiable, we may perform gradient descent 
optimization to learn the parameters.
\cite{dfc} construct a real valued, non-orthogonal matrix using a similar parameterization with the motivation
of parameter reduction by an order of magnitude on an industrial sized network. This combined with earlier
work \cite{fastfood} suggests that it is possible to create highly expressive matrices by composing simple
matrices with few parameters.

In the following section, we explain details on how to implement our model and illustrate how we bypass the
potential difficulties of working in the complex domain.

\section{Architecture details}

In this section, we describe the nonlinearity we used, how we incorporate real valued inputs 
with complex valued hidden units and map from complex hidden states to real outputs. 
{\color{red}Finally, we display some implementation considerations, including memory optimization on GPUs}.

\subsection{Complex hidden units}

Our implementation represents all complex numbers using real values in terms of their
real and imaginary parts. Under this framework, we sidestep the lack of support for complex numbers 
by most deep learning frameworks. Consider multiplying the complex weight matrix 
$\matr{W} = \matr{A} + i \matr{B}$ by the complex hidden vector $h = x + i y$, where
$\matr{A}, \matr{B}, x, y$ are real.
It is trivially true that $\matr{W}h = (\matr{A}x - \matr{B}y) + i (\matr{A}y + \matr{B}x)$.
When we represent $v \in \CC^n$ as {\color{red} (this is not pretty)} $\big(Re(v)^\top, Im(v)^\top \big)^\top \in \RR^{2n}$ , we
compute complex matrix vector products with real numbers as follows

\begin{equation} \begin{pmatrix} Re(\matr{W}h) \\ Im(\matr{W}h) \end{pmatrix}  
= \begin{pmatrix} \matr{A} & -\matr{B} \\ \matr{B} & \ \ \ 
\matr{A} \end{pmatrix} \begin{pmatrix} Re(h) \\ Im(h) \end{pmatrix} .
\end{equation}

More generally, let $f: \CC^n \rightarrow \CC^n$ be any complex function and $z = x + i y$ 
any complex vector. We may write $ f(z) = \alpha(x, y) + i \beta(x, y) $ where 
$\alpha, \beta : \RR^n \rightarrow \RR^n$. 
This allows us to implement everything using real valued operations, compatible with any
any deep learning framework with automatic differentiation such as Theano \citep{Fred2010}.

\subsection{Input to hidden and the nonlinearity}

As is the case with most recurrent networks, our uRNN follows the same hidden to hidden mapping as 
equations \ref{linoutput} and \ref{nonlinoutput} with $\matr{V}_t = \matr{V}$ and $\matr{W}_t = \matr{W}$. 
Denote the size of the complex valued hidden states as $n_h$.
The input to hidden matrix is complex valued, $\matr{V} \in \CC^{n_h \times n_h}$. 
We learn the initial hidden state $h_0 \in \CC^{n_h}$ as a parameter of the model also.

Choosing an appropriate nonlinearity is not trivial in the complex domain.
As discussed in the introduction, using a ReLU is a natural choice in combination with a norm preserving
weight matrix. {\color{red}Previous work on neural networks with complex valued} hidden states 
have applied ReLU nonlinearities on the real and imaginary parts separately \ref{Old CRNN papers}.
However, we found that such a nonlinearity {\color{red}tended to lead to} poor performance.
Our intuition as to why applying separate ReLU nonlinearities to the real 
and imaginary parts performed badly is that such an operation brutally impacts the 
phase of a complex number.

We speculate that maintaining the phase of hidden states may be important for storing information 
across a large number of time steps, and our experimentation supported this claim.
A variation of the ReLU that we deem modReLU, is what we 
finally chose as our nonlinearity. The modReLU is a pointwise nonlinearity,  
$\sigma_\mathrm{modReLU} (z) : \CC \rightarrow \CC$, which
affects only the absolute value of a complex number, and we define it as 
\begin{equation} \sigma_\mathrm{modReLU} (z) = 
\left\{
  \begin{array}{ll}
    (|z|+b|) \frac{z}{|z|}  & \mbox{if } |z| + b \geq 0 \\
    0 & \mbox{if } |z| + b < 0
  \end{array}
\right.
\end{equation}

where $b \in \RR$ is a bias parameter of the nonlinearity. For a $n_h$ dimensional hidden space
we learn $n_h$ nonlinearity bias parameters, one per dimension. 
Note that the modReLU is similar to the ReLU in spirit, in fact more concretely
$\sigma_\mathrm{modReLU}(z) = \sigma_\mathrm{ReLU}(|z| + b) \frac{z}{|z|}$. 

\subsection{Hidden to output}

To map hidden states to output, we define a matrix $\matr{U} \in \RR^{2n_h \times n_o}$, 
where $n_o$ is the output dimension. We calculate a linear output as
\begin{equation} o_t = \matr{U} \begin{pmatrix} Re(h_t) \\ Im(h_t) \end{pmatrix} + b_o , \end{equation}

where $b_o \in \RR^{n_o}$ is the output bias. 
The linear output is real valued ($o_t \in \RR^{n_h}$) and can be used for prediction and loss function 
calculation akin to typical neural networks (e.g. it may be passed through a softmax which is used for 
cross entropy calculation for classification tasks).


\subsection{Initialization}

Due to the stability of the norm preserving operations of our network, we found that performance was
not very sensitive to initialization of parameters.
For full disclosure and reproducibility, we explain our initialization strategy for each parameter below.

\begin{itemize}
  \item We initialize $\matr{V}$ and $\matr{U}$ (the input and output matrices) as in \cite{Glorotinit},
  with weights sampled independently from uniforms, $\mathcal{U}\left[-\frac{\sqrt{6}}{\sqrt{n_{in}+ n_{out}}}, \frac{\sqrt{6}}{\sqrt{n_{in}+ n_{out}}}\right]$.
  \item The biases, $b$ and $b_o$ are initialized to 0. The consequence of this is that at initialization, 
    the network is linear with unitary weights, which seems to help early optimization \citep{Saxe2014}.
  \item The reflection vectors for $\matr{R}_1$ and $\matr{R}_2$ are initialized coordinate-wise from a 
  uniform $\mathcal{U}[-1, 1]$. Note that the reflection matrices are invariant to scalar multiplication 
  of the parameter vector, hence the width of the uniform initialization is unimportant.
  \item The diagonal weights for $\matr{D}_1, \matr{D}_2$ and $\matr{D}_3$ are sampled 
  from a uniform, $\mathcal{U}[-\pi, \pi]$. This ensures that the diagonal entries $\matr{D}_{j,j}$
  are sampled uniformly over the complex unit circle.
  \item We initialize $h_0$ with a uniform, 
  $\mathcal{U}\left[-\sqrt{\frac{3}{2n_h}}, \sqrt{\frac{3}{2n_h}} \right]$, 
  which results in $\mathbb{E}\left[\|h_0\|^2\right] = 1$. Since the norm of the hidden units are roughly 
  preserved through unitary evolution and inputs are typically whitened to have norm 1, 
  we have hidden states, inputs and linear outputs of the same order of magnitude, which seems to
  help optimization \ref{batchnorm?}.
\end{itemize}

\section{Experiments}

In this section we explore the performance of our uRNN in relation to RNN with tanh activations,
IRNN \citep{Quoc2015}, an RNN with ReLU activations with weight matrix initialized to the identity, 
and LSTM \citep{LSTM} models. We show that the uRNN shines quantitatively when it comes to modelling 
long term dependencies and exhibits qualitatively different learning properties to the other models 
 
We chose a handful of tasks to evaluate the performance of the various models.
The tasks were especially created to be be pathologically hard, and have been used 
as benchmarks for testing the ability of a model to capture long-term memory \citep{LSTM} \citep{Quoc2015} \citep{NTM}
\citep{HF}

We tried a handful of optimization algorithms for the various models we wished to compare. 
RMSProp optimization \citep{RMSPROP} tended to lead to fastest convergence and is what we stuck to for all
experiments here on in. However, we found that the IRNN was particularly unstable, and would only run without 
blowing up with incredibly low learning rates and gradient clipping. Since the performance was so poor
relative to the other models we compare against, we do not show this model's curves in the figures. An example of the
performance using a learning rate of $10^{-8} $on one of the copy memory tasks is plotted in the Appendix. 
We used a learning rate of $10^{-3}$
and a decay rate of $0.9$. For the LSTM and RNN models, we had to clip gradients at 1 to avoid exploding 
gradients. Gradient clipping was unnecessary for the uRNN models.


\subsection{Copying memory problem}

Recurrent networks have been known to have trouble remembering information about inputs seen
many time steps previously \citep{Yoshua94, Pascanu2013}. 
We therefore want to test the uRNN's ability to recall exactly data seen a long time ago.

Following a similar setup to \cite{LSTM}, we outline the setup of the copy memory problem.
Consider 10 categories, $\{ a_i \}_{i=0}^9$.   
The input takes the form of a $T+20$ length vector of categories, where we test over a variety of values
of $T$. 
The first $10$ entries are sampled uniformly, independently and with replacement from $\{a_i\}_{i=0}^7$,
and represent the sequence which will need to be remembered. 
The next $T-1$ entries are set to $a_8$, which can be thought of as the 'blank' category. 
The next single entry is $a_9$, which represents a delimiter, which should indicate to the algorithm
that it is required to reproduce the initial $10$ categories in the output. 
The remaining $10$ entries are set to $a_8$. The required output sequence consists of $T+10$ 
repeated entries of $a_8$, followed by the first $10$ categories of the input sequence in exactly the
same order. The goal is to minimize the average cross entropy of category predictions 
at each time step of the sequence.
 
The task amounts to having to remember a categorical sequence of length 10, for $T$ time steps.

A simple baseline can be established by considering an optimal strategy which does not require any memory, 
which we deem a \textit{memoryless} strategy. The memoryless strategy would be to predict $a_8$ for 
$T+10$ entries and then predict each of the final $10$ categories from the set $\{a_i\}_{i=0}^7$ independently
and uniformly at random. The mean categorical cross entropy of this strategy is easy to calculate, and equal 
to $\frac{10 \log(8) }{T+20} $. 

\begin{figure}[t!]
  \centering
  \label{fig7} 
  \begin{minipage}[b]{0.5\linewidth}
    %\centering
    \includegraphics[scale=0.25]{figures/memory_100.jpeg}
    \vspace{1ex}
  \end{minipage}%%
  \begin{minipage}[b]{0.5\linewidth}
    %\centering
    \includegraphics[scale=0.25]{figures/memory_200.jpeg}
    \vspace{1ex}
  \end{minipage} 
  \begin{minipage}[b]{0.5\linewidth}
    %\centering
    \includegraphics[scale=0.25]{figures/memory_300.jpeg}
    %\vspace{4ex}
  \end{minipage}%% 
  \begin{minipage}[b]{0.5\linewidth}
    %\centering
    \includegraphics[scale=0.25]{figures/memory_500.jpeg}
    %\vspace{4ex}
  \end{minipage} 
  \caption{Results of the copying memory problem for time lag $=100, 200, 300, 500$.}
\end{figure}

We ran experiments where the RNN with tanh activations, IRNN and LSTM had hidden layers of size 32 and the
uRNN had a hidden layer of size 128. This equates to roughly 1000 parameters for the RNN and IRNN, roughly 
1900 parameters for the uRNN and roughly 4000 parameters for the LSTM. Note that the number of uRNN parameters
is less than half of that of the LSTM.

In figure 1, we see that except in the simplest case both the RNN with tanh and more surprisingly 
the LSTMs get almost exactly the same cost as the memoryless strategy. This behaviour is consistent 
with the results of \cite{NTM}, in which poor performance is reported for the LSTM for a very similar 
copying memory problem.

The uRNN consistently achieves perfect performance in relatively few iterations, even when having to recall
sequences after 500 time steps. What is remarkable is that the uRNN does not get stuck at the baseline at 
all, whilst the LSTM and RNN do. This behaviour suggests that the representations learned by the uRNN
have very different properties from both the LSTM and classical RNNs. 

\subsection{Adding Problem}
In this problem, the input consists of two sequences of length $T$. The first sequence (let's call it $x$) consists of numbers sampled uniformly at random from a uniform $\mathcal{U}[0,1]$. The second one consists of all zeros, except for two positions that are 1. This two positions (let's call them $i$ and $j$) are sampled at random from the first and second half respectively. An output is given only after the last time step, and it is $x_i + x_j$ (i.e. the numbers from the first sequence in the marked positions).

With some elemntary calculations, we can see that predicting 1 regardless of the input sequence gives in expectation 0.167 of error, so we use this as a baseline.

\begin{figure}[ht] 
  \label{ fig7} 
  \begin{minipage}[b]{0.5\linewidth}
    \centering
    \includegraphics[scale=0.25]{figures/memory_100.jpeg}
    \vspace{4ex}
  \end{minipage}%%
  \begin{minipage}[b]{0.5\linewidth}
    \centering
    \includegraphics[scale=0.25]{figures/adding_200.jpeg}
    \vspace{4ex}
    \end{minipage} 
  \begin{minipage}[b]{0.5\linewidth}
    \centering
    \includegraphics[scale=0.25]{figures/adding_400.jpeg}
    \vspace{4ex}
    \end{minipage}%% 
  \begin{minipage}[b]{0.5\linewidth}
    \centering
    \includegraphics[scale=0.25]{figures/adding_750.jpeg}
    \vspace{4ex}
  \end{minipage} 
  \caption{Results of the adding problem for $T=100, 200, 400, 750$.}
\end{figure}

For this problem we ran the same models as before, with 128 hidden units on RNN with tanh, IRNN and LSTM, and 512 for the uRNN. This means that the RNN and IRNN have roughly 10K parameters, the LSTM about 40K, and the uRNN near 15K. We trained all models on batches of size 20 and 50, the best results are reported. The RNN with tanh activation was not shown for 750 so the differences between the LSTM and the uRNN curves are clearer, and the model achieved almost constant baseline as expected. 

We can see that up to $T=400$ both the LSTM and the uRNN learn models that go well below the baseline. In the case of sequences of size 750, both algorithms do quite well, with errors remarkably below the baseline, although not quite reaching 0. The uRNN achieves lower error, but it's curve is more noisy.

The RNN with tanh and IRNN weren't able to go below baseline on any of the experiments. \cite{Quoc2015} reports that their RNN solves the problem for $T=150$ and the IRNN for $T=300$, with more than a million iterations before they start learning. However, none of the algorithms was able to learn for $T=400$, a task on which both our uRNN and the LSTM do quite well. We evidenced two big differences with the results reported by \cite{Quoc2015} on this task. The first is the performance of the IRNN, which wasn't able to converge in our case for any experiment. The second, the overwhelming improvement of our LSTM in comparison to theirs. Both of these differences can potentially be explained by our use of RMSProp instead of momentum.

{\color{red} Show that the LSTM learns to predict only the number for the last half on $T=750$. Plot the correct figure for $T=100$}

\subsection{Pixel-by-pixel MNIST}
IRNN

\subsection{Optimization experiments}
Ian Good

\bibliography{uRNN}
\bibliographystyle{uRNN}

\end{document}
